\title{Assignment 3}
\author{Justin Espiritu}
\documentclass[letterpaper]{article}
\usepackage[utf8]{inputenc}
\usepackage{amsmath}
\usepackage{wasysym}
\usepackage{amssymb}
\usepackage{commath}
\usepackage{fancyhdr}
\usepackage[a4paper, margin=1in]{geometry}
\pagenumbering{gobble}
\renewcommand{\thesection}{\arabic{section}.}
\renewcommand{\thesubsection}{\alph{subsection})}
\renewcommand{\thesubsubsection}{Case \arabic{subsubsection}.}

\pagestyle{fancy}
\fancyhf{}
\lhead{Justin Espiritu}
\rhead{Assignment 3}
\begin{document}
	\section{}
    	\subsection{}
        	\begin{displaymath}
        		\begin{array}{ll}
        			\exists \neg n P(2n^2 + 3n + 1) & $Prove the negation$\\
                    \neg \exists n P(2n^2 + 3n + 1) & \\
                  	\forall n \neg P(2n^2 + 3n + 1) & $De Morgan's$\\
                    x = $ an arbitrary integer$ & $Assumption$\\
                    \neg P((x)^2 + 3(x) + 1 & $Universal instantiation$\\
                    \neg P((2x + 1)(x + 1) & $Distribution$\\
					\neg P((2x + 1)(x + 1) & $because the equation can be a factor of 2 numbers it is not prime$\\
                    \hline
                    \therefore \exists P(2n^2 + 3n + 1) & $Invalid (Proof by contradiction)$\\
        		\end{array}
        	\end{displaymath}
            Because the negation is proved to be true the hypothesis is invalid.
		\subsection{}
        	\begin{displaymath}
            	\begin{array}{ll}
            		x = 0 & $assumption$\\
                    P(2(0)^2 + 3(0) + 2) &\\
                    P(2) &\\
                    2 $ is prime$ & \\
                    \hline
                    \therefore \exists P(2n^2 + 3n + 1) & $Valid (Definition of $\exists$)$\\
            	\end{array}
        	\end{displaymath}
	\section{}
    	\begin{displaymath}
    		\begin{array}{rcl}
    			\frac{a}{b} = $some rational number where a and b are integers$ && 
                	$assumption/definition of a rational number$\\
                \frac{n}{m} = $some rational number where a and b are integers$ && 
                	$assumption/definition of a rational number$\\
                x = $ some irrational number $ && $assumption$\\
                \frac{a}{b} - x &=& \frac{n}{m}\\
                -x &=& \frac{n}{m} - \frac{a}{b}\\
                x &=& \frac{a}{b} - \frac{n}{m}\\
                x &=& \frac{am - nb}{bm}\\
                \frac{am - nb}{bm} && $by definition is a rational number$\\
                \hline
                \therefore \frac{a}{b} - x && $cannot be irrational via contradiction above$\\ 
    		\end{array}
    	\end{displaymath}
	\section{}
    	\begin{displaymath}
    		\begin{array}{rll}
            	\abs{x} + \abs{y} \geq \abs{x + y} && $assumption$\\
    			\abs{n} &=& \binom{n \geq 0 \rightarrow n}{n < 0 \rightarrow (-n)} 
    		\end{array}
    	\end{displaymath}
        \subsubsection{}
        	\begin{displaymath}
        		\begin{array}{ll}
        			x, y \geq 0 \Rightarrow \abs{x} + \abs{y}: x + y & $definition of absolute value$\\
                    x, y \geq 0 \Rightarrow x + y \geq 0 &\\
                    x + y = \abs{x + y} & $definition of absolute value$\\
                    \hline
                    \therefore \abs{x} + \abs{y} = \abs{x + y} & $when $ x $ and $ y \geq 0\\
        		\end{array}
        	\end{displaymath}
        \subsubsection{}
        	\begin{displaymath}
        		\begin{array}{ll}
        			x, y < 0 \Rightarrow \abs{x} + \abs{y}: (-x) + (-y) & $definition of absolute value$\\
                    (-x) + (-y) = -(x + y) &\\
                    x, y < 0 \Rightarrow -(x + y) < 0 &\\ 
                    -(x + y) = \abs{x + y} & $definition of absolute value$\\
                    \hline
                    \therefore \abs{x} + \abs{y} = \abs{x + y} & $when $ x $ and $ y < 0\\
        		\end{array}
        	\end{displaymath}
        \subsubsection{}
        	\begin{displaymath}
        		\begin{array}{ll}
                	x \geq 0, y < 0 \Rightarrow \abs{x} + \abs{y}: x + (-y) & $definition of absolute value$\\
                	y < 0 \Rightarrow x + (-y) > x & $because we are subtracting a negative number it makes x bigger$\\
                    \hline
                	\abs{x + y} $ can either be $ x + y \geq 0 $ or $ -(x + y) < 0 & $can do this via definition of absolute value$\\
                    x \geq 0, y < 0 \Rightarrow x + y < x & $because $y < 0, x + $a negative number will be less than x$\\
                    x + (-y) > x\Rightarrow x +(-y) > x + y & $ because $x + y < x $ we can say $x +(-y) > x + y \\
                    \hline
                    -(x + y) = -x - y & $ distribution$\\
                    x \geq 0 \Rightarrow x > -x & \\
                    x + (-y) = x - y \Rightarrow x - y > -x - y & \\
                    \hline
                    \therefore \abs{x} + \abs{y} > \abs{x + y} & $ when $ x \geq 0, y < 0\\
                \end{array} 
        	\end{displaymath}
       \subsubsection{}
        	\begin{displaymath}
        		\begin{array}{ll}
                	x < 0, y \geq 0 \Rightarrow \abs{x} + \abs{y}: (-x) + y & $definition of absolute value$\\
                	x < 0 \Rightarrow (-x) + y > y & $because we are adding a negative, negative number it makes y bigger$\\
                    \hline
                	\abs{x + y} $ can either be $ x + y \geq 0 $ or $ -(x + y) < 0 & $can do this via definition of absolute value$\\
                    x < 0, y \geq 0 \Rightarrow x + y < y & $because x $< 0, y + $a negative number will be less than x$\\
                    (-x) + y > y\Rightarrow (-x) + y > x + y & $ because $x + y < y $ we can say $ (-x) + y > x + y \\
                    \hline
                    -(x + y) = -x - y & $ distribution$\\
                    y \geq 0, \Rightarrow y > -y &\\
                    (-x) + y > -x - y &  $because $ y > -y $ adding -x to y will always be greater than (-x) - y, as long as $ y \geq 0\\
                    \hline
                    \therefore \abs{x} + \abs{y} > \abs{x + y} & $ when $ x \geq 0, y < 0\\
                \end{array} 
        	\end{displaymath}
    Because all the cases of $\abs{x} + \abs{y}$ are either $\geq$ or $=$, $\abs{x} + \abs{y} \geq \abs{x + y}$\\
	\section{}
    	\begin{displaymath}
    		\begin{array}{rll}
    			E(x) &=& x $ is even$\\
                O(x) &=& x $ is odd$\\
                E(x) &=& \neg O(x)\\
                O(x) &=& \neg E(x)\\
                E(x^2 + x + 1) \rightarrow O(x) &&\\
                \neg O(x) \rightarrow \neg E(x^2 + x + 1) && $start proof by contrapositive$\\
                E(x) \rightarrow O(x^2 + x + 1) && $definition$\\
                k = $ some arbitrary int$ && $assumption$\\
                2k = $ even $ 2k + 1 = $ odd $ && $definition of odd and even$\\
               	2k^2 + 2k + 1 = (2k^2 + 2k) + 1 &&\\
                2k^2 + 2k = $ some integer $ + 1 = 2k + 1 && $definition of odd$\\
                \hline
                \therefore E(x^2 + x + 1) \rightarrow O(x) && $proof by contrapositive$\\
    		\end{array}
    	\end{displaymath}
	\section{}
    	\begin{displaymath}
    		\begin{array}{ll}
            	3a + 5b = 7m $, where m is some integer$ & 3c + 5d = 7n $, where n is some integer$\\
                3(a + c) + 5(b + d) = 7z & z = $some integer$\\
                3a + 3c + 5b + 5d = 7z & $distribute$\\
                3a + 3b + 3c + 5d = 7z & \\
                7m + 7n = 7z & $definition$\\
                7(m + n) = 7z & $distribution$\\
                m + n = $some integer$ & $integer + integer = some integer$\\
                7($some integer$) = 7 ($some integer$) &\\ 
                \hline
                \therefore 3(a + c) + 5(b+d) = 7z & $valid$\\
    		\end{array}
    	\end{displaymath}
\end{document}
