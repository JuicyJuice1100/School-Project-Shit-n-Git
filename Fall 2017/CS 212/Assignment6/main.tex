\title{Assignment 6}
\author{Justin Espiritu}
\documentclass[letterpaper]{article}
\usepackage[utf8]{inputenc}
\usepackage{amsmath}
\usepackage{wasysym}
\usepackage{amssymb}
\usepackage{commath}
\usepackage{fancyhdr}
\usepackage[a4paper, margin=1in]{geometry}
\pagenumbering{gobble}
\renewcommand{\thesection}{\arabic{section}.}
\renewcommand{\thesubsection}{\alph{subsection})}

\pagestyle{fancy}
\fancyhf{}
\lhead{Justin Espiritu}
\rhead{Assignment 6}
\begin{document}
	\section{}
    	\begin{displaymath}
            \begin{array}{llll}
            $Type$ & $Vanilla Induction$ &&\\
            $Prove:$ & 10^n$ mod $3=1 $ $ \forall_n \geq 0 &&\\
            $Base:$ & n=0 &&\\
            & 10^0 $ mod $ 3 = 1 &&\\
            $Inductive Step:$ & $Assume $ 10^k $ mod $ 3 = 1, $ then $ 10^{k+1} $ mod $ 3 = 1 &&\\
            \\
            & 10^{k+1} $ mod $ 3 = (10^k $ mod $ 3 * 10^1 $ mod $ 3) $ mod $ 3 && \\
            \\
            & $using Theorem from 6.11.12, it is applicable here as property of log can simplify to$&&\\
            & 10^k * 10^1 $. We can then apply the Theorem from 6.11.12 which become the above equation$&&\\
            \\
            & (10^k $ mod $ 3 * 1) $ mod $ 3 && \\
            & (1 * 1) $ mod $ 3 && \\
            \\
            & $we can simplify $ 10^k $ mod $ 3 $ to 1 via the assumption$ &&\\
            \\
            & 1 $ mod $ 3 = 1 &&\\
            \\
            & $ we have gotten the original assumption proving it is true$ && \\
            \end{array}
        \end{displaymath}
    \section{}
    	\begin{displaymath}
            \begin{array}{llll}
            $Type$ & $Structural Induction$ &&\\
            $Prove:$ & h={\lfloor}\log_2{n}{\rfloor} &&\\
            & $h = height n = number of nodes$ &&\\
            & $all logs are base 2$ &&\\
            $Base:$ & n=1 &&\\
            $Inductive Step:$ & $Assume $ T = h $ then $ T_1 $ is at most $ h+1, $ otherwise it's h$ &&\\
            & $where T is the height of the subtree inside the original tree $&&\\
            \\
            & $and $ T_1 $ is the height of the the complete tree. $&&\\
			& T $ is a perfect tree via definition of a complete tree $ &&\\
            & n = $ amount of nodes$ &&\\
            & n = 2^{h+1} - 1 $via definition of a perfect tree $ &&\\
            & n + 1 = 2^{h+1} \rightarrow\log{(n+1)}=h+1\rightarrow h=\log{(n+1)}-1&&\\
			\\
            & $ solve for h using properties of log $ &&\\
            \\
            & h=\log{(n+1)}-1 \rightarrow \log{((n+1)/2)} &&\\
            \\
            & $ via property of logs $ &&\\
            \\
            & $plug in for $ T_1 \rightarrow \log{((n+1)/2)} - 1 &&\\
            & \log((n+1)/2/2) \rightarrow \log(n+1) $ via property of logs$ &&\\
            &$ I don't know where to go from here... $&&\\
            &$ but I need to end up where $T_1$ returns to the proof.$&&\\
            \end{array}
        \end{displaymath}
    \section{}
    	\begin{displaymath}
            \begin{array}{llll}
            $Type$ & $Strong Induction$ &&\\
            $Prove:$ 3 + 2^n &&\\
            $Base:$ & n = 3 &&\\
            & a_3 = 3(a_2) - 2(a_1) &&\\
            & 3(5) - 2(7) = 11 &&\\
            & 3 + 2^3 = 11 &&\\
            & 10^0 $ mod $ 3 = 1 &&\\
            $Inductive Step:$ & $Assume $ a_n = 3 + 2^n $ $\forall_n \leq k &&\\
            & $then $3 + 2^{k+1} $must be true $ &&\\
            \\
          	& a_{k+1} = 3a_{k+1-1} - 2a_{k+1-2} &&\\
            \\
            & $plug in $ k+1 &&\\
            \\
            & a_{k+1} = 3a_{k} - 2a_{k-1} &&\\
            \\
            & $simplify$&&\\
            \\
            & = 3(3 + 2^k) - 2(3+2^{k-1}) &&\\
            \\
            & $ via assumption $ &&\\
            \\
            & = 9 + 3(2^k) - 6 - 2(2^{k-1}) &&\\
            \\
            & $ distribution $ &&\\
            \\
            & 3 + 3(2^k) - 2(2^{k-1}) &&\\
            \\
            & $simplify$ &&\\
            \\
            & (3 * 2 - 2)2^{k-1} + 3 &&\\
            \\
            & $factor out a $ 2^{k-1} &&\\
            \\
            & 2^2 * 2^{k-1} + 3 &&\\
            \\
            & $ evaluate parenthesis which is 4, then make it base 2, which is $ 2^2 &&\\
            \\ 
            & 2^{k+1} + 3 &&\\
            \\
            & $ simplify by adding powers together of same base.  $&&\\
            & $ This returns what we are trying to prove making this valid $ &&\\
            \end{array}
        \end{displaymath}
    \section{}
    	No, the basis step is incorrect and the step in induction where you state
        "via induction assumption".  Cases in this proof are not all the same which 
        means that the basis case must be stronger either one more base.  In the
        induction step you assume $r^{k-1}$ is always one 1 which it is not. 
        If k = 0 it does not work and that still follows your assumption.
\end{document}
