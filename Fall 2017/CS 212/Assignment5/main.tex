\title{Assignment 5}
\author{Justin Espiritu}
\documentclass[letterpaper]{article}
\usepackage[utf8]{inputenc}
\usepackage{amsmath}
\usepackage{wasysym}
\usepackage{amssymb}
\usepackage{commath}
\usepackage{fancyhdr}
\usepackage[a4paper, margin=1in]{geometry}
\pagenumbering{gobble}
\renewcommand{\thesection}{\arabic{section}.}
\renewcommand{\thesubsection}{\alph{subsection})}

\pagestyle{fancy}
\fancyhf{}
\lhead{Justin Espiritu}
\rhead{Assignment 5}
\begin{document}
	\section{}
    	Given the public key (7, 143) $N = 143$, $e = 7$\\
        First found 2 prime numbers multiplied that equal, 143.  In this case 11 and 13.\\
        Then had to find $\phi$, which is $(11-1)(13-1) = 120, \phi = 120$\\
		Given that the inverse is between 100 and 105, I narrowed it down to 101 and 103 as they are the only prime numbers out of those.\\
        $101 * e = 707 \Rightarrow 707$ mod $120 = 107$, 101 is not an inverse.\\
        $103 * e = 712 \Rightarrow 712$ mod $120 = 1$, 103 is an inverse.\\
        So now we have $N=143$, $e = 7$, $phi = 120$, $d = 103$ Cipher or C = 106\\
        decryption $= C^d$ mod $143 = 106^103$ mod $143 = 24$\\
        Therefore the decrypted message is 24.
    \section{}
    	\subsection{}
        	yes, the function performs at $O(nlog(n))$ or $O(n^3)$ which $O(nlog(n))$ still fits into the ceiling of $O(nlog(n))$
        \subsection{}
        	no, the function performs at $\Omega(nlog(n))$ or $\Omega(n^3)$ which both do not satisfy $\Omega(nlog(n))$ therefore it is no
        \subsection{}
        	no, because the function performs at $O(nlog(n))$ or $O(n^3)$ which $O(nlog(n))$ does is not $O(n^3)$ as it's below that ceiling
        \subsection{}
        	yes, the function performs $\Omega(nlog(n))$ or $\Omega(n^3)$ which both fit into the floor of $\Omega(n^3)$, which satisfies both
    \section{}
    	\subsection{}
            \begin{tabular}{ccc}
                arithmetic & s & m\\
                m = a[0] & undefined & 6\\
                s = 0 & 0 & 6\\
                s = s + a[j] & 6 & 6\\
                s = s + a[j] & -1 & 6\\
                s = s + a[j] & 7 & 6\\
                m = s & 7 & 7\\
                s = s + a[j] & 10 & 7\\
                m = s & 10 & 10\\
                s = s + a[j] & 19 & 10\\
                m = s & 19 & 19\\
                s = s + a[j] & 7 & 19\\
                s = 0 & 0 & 19\\
                s = s + a[j] & -7 & 19\\
                s = s + a[j] & 1 & 19\\
                s = s + a[j] & 4 & 19\\
                s = s + a[j] & 13 & 19\\
                s = s + a[j] & 1 & 19\\
                s = 0 & 0 & 19\\
                s = s + a[j] & 8 & 19\\
                s = s + a[j] & 11 & 19\\
                s = s + a[j] & 20 & 19\\
                m = s & 20 & 20\\
                s = s + a[j] & 8 & 20\\
                s = 0 & 0 & 20\\
                s = s + a[j] & 3 & 20\\
                s = s + a[j] & 11 & 20\\
                s = s + a[j] & -1 & 20\\
                s = 0 & 0 & 20\\
                s = s + a[j] & 9 & 20\\
                s = s + a[j] & -3 & 20\\
                s = 0 & 0 & 20\\
                s = s + a[j] & -12 & 20\\
            \end{tabular}\\
            final value of m = 20
        \subsection{}
        	21 times
        \subsection{}
            \begin{tabular}{cc}
            algorithm & $\Theta$ \\
            m = a[0] & 1\\
            i loop & $n$\\
            s = 0 & $n * 1$\\
            j loop & $n * (n - i)$\\
            s = s + a[j] & $n * (n - i) * 1$\\
            s > m & $n * (n - i) * 1$\\
            m = s & $n * (n - i) * 1$\\
            \end{tabular}\\
            $\Theta(n^2)$, shown by the table above.  J loop will happen n-i times which also includes everything, inside of it will also happen n-i times.  We then multiply the individual $\Theta$ times with the times it happens according to the loop they are in.  We will then add all the $\Theta$ times together.  From there we can simplify to the highest $\Theta$, which in this case is $n * (n - i) * 1$ which then can be simplified to just $\Theta(n^2)$.
    \section{}	
    	\subsection{}
        	\begin{tabular}{ccc}
        		algorithm & s & m\\
                m = a[0] & undefined & 6\\
                s = 0 & 0 & 6\\
                s = s + a[i] & 6 & 6\\
                s = s + a[i] & -7 & 6\\
                s = 0 & 0 & 6\\
                s = s + a[i] & 8 & 6\\
                m = s & 8 & 8\\
                s = s + a[i] & 11 & 8\\
                m = s & 11 & 11\\
                s = s + a[i] & 20 & 8\\
                m = s & 20 & 20\\
                s = s + a[i] & 8 & 20\\
        	\end{tabular}\\
            final value of m = 20
        \subsection{}
        	\begin{tabular}{cc}
            	algorithm & $\Theta$ \\
            	m = a[0] & 1\\
            	i loop & $n$\\
            	s = 0 & $n * 1$\\
            	s = s + a[j] & $n * 1$\\
            	s $>$ m & $n * 1$\\
                m = s & $n * 1$\\
                s $<$ 0 & $n * 1$\\
                s = 0 & $n * 1$\\
            \end{tabular}\\
            $\Theta(n)$ , shown by the table above.  Similar to problem 3, m = a[0] will happen at $\Theta(1)$, everything inside the loop will happen n times their theta, which in this case everything inside performs at $\Theta(1)$.  This results in the highest $\Theta$ to be $n * 1$ which can be simplified to $\Theta(n)$
\end{document}