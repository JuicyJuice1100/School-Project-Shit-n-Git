\documentclass[12pt]{article}

\usepackage{amsmath,amssymb,latexsym,enumitem,comment,color,ifthen}

\setlength{\oddsidemargin}{-0.5in}
\setlength{\evensidemargin}{-0.5in}
\setlength{\textwidth}{7.5in}
\setlength{\topmargin}{0in}
\setlength{\textheight}{8.5in}

\newcommand{\qed}{\hfill$\square$}
\newcommand{\assignNum}{2}
\newcommand{\numPoints}{45}
\newcommand{\dueDate}{Monday 9/17/2018}
\newcommand{\pbStatement}[2]{
  \colorbox{yellow}{\parbox[t]{0.93\textwidth}
        {\bf (#1 \ifthenelse{#1 > 1}{points}{point})
 #2}}}
\newcommand{\fillIn}[1]{\fbox{\parbox[t]{0.93\textwidth}{\it #1}}}
\newcommand{\header}{
\noindent{\textit{CS 321 - Algorithms - Fall 2018\hfill \yourName}}
\begin{center}
  \textbf{\Large Assignment \assignNum}\\\colorbox{red}{\Large\bf
     \textcolor{white}{Due BEFORE 8:00AM on  \dueDate}}\\
  On time /  20\% off / no credit\\\textbf{Total points: \numPoints}
  \bigskip

  \hrule\medskip You are allowed to work with a partner on this
  assignment. If you decide to form a pair, make sure to include
  both names above, but submit only one file to D2L.
  \medskip \hrule \end{center}

\medskip
This assignment will test your knowledge of asymptotic notation.  You
must write up your solutions to this assignment IN THIS FILE using
\LaTeX\ by filling in all of the boxes below. If your submitted
\texttt{.tex} file does not compile, then you will receive 0 points.

All of the proofs that you write for this and subsequent assignments should be
{\bf as detailed as possible}.  You must write EACH step on its own line so that
it is very easy to see how each step follows directly from the previous step.
Furthermore, be sure to explain/justify each step in your proof. For example,
state that you added 5 to each side, that you simplified a fraction, etc. Do NOT
skip a bunch of algebraic steps in your proofs. For example, do not simply say
$\frac{k(k+1)}{2} + (k + 1) = \frac{(k+1)(k+2)}{2}$. It is a true statement, but
it is not trivial. It is not obvious that the left hand side equals the right
hand side. If you are unsure whether a step is needed or not, ask me first
before omitting it. Use the symbol $\square$ (i.e., \texttt{$\backslash$qed} in
your \texttt{.tex} file) to denote the end of each proof.

Here is an example of proper justification of algebraic steps and
formatting of a sequence
of equations:
\begin{align*}
  \frac{k(k+1)}{2} + (k + 1) & = \frac{k(k+1)}{2} + \frac{2(k + 1)}{2}
                                       & \mathrm{unified\ denominators}\\
                             & = \frac{k(k+1)+2(k+1)}{2}
                                       & \mathrm{added\ fractions}\\
                             & = \frac{(k+1)(k+2)}{2}
                             & \mathrm{factored\ out\ common\ factor}\\
\end{align*}\vspace*{-10mm}

Note how this list of equations is aligned vertically into three columns around
the equal signs and with the justifications on the right. This is the REQUIRED
format for your submissions whenever applicable.

You should NOT add any \LaTeX\ packages to your \texttt{.tex} file.

\medskip
\textbf{Submission procedure:}
\begin{enumerate}[itemsep=-2mm]

\item Complete this file, called \texttt{a\assignNum .tex}, with your full name
and answers typed up below.

\item Compile this file to produce a file called \texttt{a\assignNum .pdf}. Make
sure that this file compiles properly and that its contents and appearance meet
the requirements described in this handout.

\item Create a directory called \texttt{a\assignNum} and copy exactly two files
into this directory, namely:\vspace*{-4mm}

  \begin{itemize}[itemsep=-1mm]

  \item \texttt{a\assignNum .tex} (this file with all of your answers and
  name(s) added)

  \item \texttt{a\assignNum .pdf} (the compiled version of the file above)
  \end{itemize}

\item Zip up this directory to yield a file called \texttt{a\assignNum .zip}

\item Submit this zip file to the D2L dropbox for A\assignNum\  before the
deadline above.

\item Submit a single-sided, hard copy of your \texttt{a\assignNum .pdf} file
BEFORE the beginning of class on the due date above.
\end{enumerate}
}

%************************************
% No need to modify anything above this line
% but DO fill this in!

\newcommand{\yourName}{\fbox{Your name(s) go here}}

%************************************g

\begin{document}
\header
\textbf{Problem statements}
\begin{enumerate}

  %%%%%%%%%%%%%%%%%%%%%%%%%%%%%%%%%%%%%%%%%%%%%%%%%%%%%%%%%%%%%%%%%%%%%%%%%%%
   \item                            % Problem 1
  %%%%%%%%%%%%%%%%%%%%%%%%%%%%%%%%%%%%%%%%%%%%%%%%%%%%%%%%%%%%%%%%%%%%%%%%%%%
     \pbStatement{5}{Prove $10\cdot\log_2 N^{100} = O(\log_{16}
     N)$. Your proof MUST use the definition on slide 2-4. For
     full credit, your proof must use the smallest possible value for $N_0$.}

   {\bf Proof:}

  %%%%%%%%%%%%%%%%%%%%%%%%%%%%%%%%%%%%%%%%%%%%%%%%%%%%%%%%%%%%%%%%%%%%%%%%%%%
  \fillIn{Your proof goes here} % replace this line with your proof
  %%%%%%%%%%%%%%%%%%%%%%%%%%%%%%%%%%%%%%%%%%%%%%%%%%%%%%%%%%%%%%%%%%%%%%%%%%%

  %%%%%%%%%%%%%%%%%%%%%%%%%%%%%%%%%%%%%%%%%%%%%%%%%%%%%%%%%%%%%%%%%%%%%%%%%%%
   \item                            % Problem 2
  %%%%%%%%%%%%%%%%%%%%%%%%%%%%%%%%%%%%%%%%%%%%%%%%%%%%%%%%%%%%%%%%%%%%%%%%%%%
     \pbStatement{10}{Prove that $12N^2 - 36N + 24 = \Theta(N^2)$. Your proof
     MUST use:
     \begin{enumerate}
       \item the definition of the $\Theta(.)$ notation on slide 2-8 (and
       thus also those on slides 2-4 and 2-6), and
       \item  the constants defined at the bottom of page 46 in our text.
     \end{enumerate}
   }

   {\bf Proof:}

  %%%%%%%%%%%%%%%%%%%%%%%%%%%%%%%%%%%%%%%%%%%%%%%%%%%%%%%%%%%%%%%%%%%%%%%%%%%
  \fillIn{Your proof goes here} % replace this line with your proof
  %%%%%%%%%%%%%%%%%%%%%%%%%%%%%%%%%%%%%%%%%%%%%%%%%%%%%%%%%%%%%%%%%%%%%%%%%%%

  %%%%%%%%%%%%%%%%%%%%%%%%%%%%%%%%%%%%%%%%%%%%%%%%%%%%%%%%%%%%%%%%%%%%%%%%%%%
   \item                            % Problem 3
  %%%%%%%%%%%%%%%%%%%%%%%%%%%%%%%%%%%%%%%%%%%%%%%%%%%%%%%%%%%%%%%%%%%%%%%%%%%
     \pbStatement{10}{Prove or disprove $3^N=\Theta(N!)$. For full credit,
     your proof MUST use the definition on slide 2-8 or its negation. In other
     words, you must specify the required constant(s).}

   {\bf Proof:}

  %%%%%%%%%%%%%%%%%%%%%%%%%%%%%%%%%%%%%%%%%%%%%%%%%%%%%%%%%%%%%%%%%%%%%%%%%%%
  \fillIn{Your proof goes here} % replace this line with your proof
  %%%%%%%%%%%%%%%%%%%%%%%%%%%%%%%%%%%%%%%%%%%%%%%%%%%%%%%%%%%%%%%%%%%%%%%%%%%

%%%%%%%%%%%%%%%%%%%%%%%%%%%%%%%%%%%%%%%%%%%%%%%%%%%%%%%%%%%%%%%%%%%%%%%%%%%
 \item                            % Problem 4
%%%%%%%%%%%%%%%%%%%%%%%%%%%%%%%%%%%%%%%%%%%%%%%%%%%%%%%%%%%%%%%%%%%%%%%%%%%
   \pbStatement{10}{Prove or disprove $15^{\log_2 N}=o(N^4)$. For full credit,
   your proof MUST use the formal definition on slide 3-5 or its negation. In
   other words, you must specify the required constant(s).}

 {\bf Proof:}

%%%%%%%%%%%%%%%%%%%%%%%%%%%%%%%%%%%%%%%%%%%%%%%%%%%%%%%%%%%%%%%%%%%%%%%%%%%
\fillIn{Your proof goes here} % replace this line with your proof
%%%%%%%%%%%%%%%%%%%%%%%%%%%%%%%%%%%%%%%%%%%%%%%%%%%%%%%%%%%%%%%%%%%%%%%%%%%

 %%%%%%%%%%%%%%%%%%%%%%%%%%%%%%%%%%%%%%%%%%%%%%%%%%%%%%%%%%%%%%%%%%%%%%%%%%%
  \item                            % Problem 5
 %%%%%%%%%%%%%%%%%%%%%%%%%%%%%%%%%%%%%%%%%%%%%%%%%%%%%%%%%%%%%%%%%%%%%%%%%%%
    \pbStatement{10}{Given a list of functions, your goal is to order them
    according to their rate of growth, from smallest to largest. For example,
    if given the following functions:

    \centerline{$N^2 \hfill 10N \hfill N^2-5N+12 \hfill N \hfill
    N\log N \hfill 2^N$}

    the correct answer would be the following table:\medskip

\begin{minipage}{\linewidth}
  \center
  \[
    \begin{array}{|c|c|}\hline
      N       & 10N        \\\hline
      N\log N &            \\\hline
      N^2     & N^2-5N+12  \\\hline
      2^N     &            \\\hline
    \end{array}
  \]
\end{minipage}\medskip

    in which all functions in a given row are big-theta of each other and
    little-o of all functions in the following row (if any). The left-to-right
    order within each row is not significant.
    \bigskip

    For this problem, $\log$s are base 2 unless otherwise specified. You must
    build an  appropriately sized table that shows the correct ordering of the
    following 12 functions:\medskip

    \centerline{\hfill $N\sqrt{N} \hfill \log\log N \hfill 3^N \hfill
    N^{0.01}\log_3 N \hfill    N\log N \hfill 3^{2N}$ \hfill}

    \centerline{ \hfill  $N^{\frac{N}{\log_3 N}}  \hfill N^3 \hfill
        \sqrt{N^22^{\log\log^2 N}} \hfill 3^{N\log N^3} \hfill
        N^{\frac{3}{\log_3 N}} \hfill 3^{N+2}$ \hfill}
  }

 %%%%%%%%%%%%%%%%%%%%%%%%%%%%%%%%%%%%%%%%%%%%%%%%%%%%%%%%%%%%%%%%%%%%%%%%%%%
 \fillIn{Your table goes here}
 %%%%%%%%%%%%%%%%%%%%%%%%%%%%%%%%%%%%%%%%%%%%%%%%%%%%%%%%%%%%%%%%%%%%%%%%%%%


\end{enumerate}
\end{document}
