\documentclass[12pt]{article}

\usepackage{amsmath,amssymb,latexsym,enumitem,qtree,comment,color,ifthen}

\setlength{\oddsidemargin}{-0.5in}
\setlength{\evensidemargin}{-0.5in}
\setlength{\textwidth}{7.5in}
\setlength{\topmargin}{0in}
\setlength{\textheight}{8.5in}

\newcommand{\qed}{\hfill$\square$}
\newcommand{\assignNum}{1}
\newcommand{\dueDate}{Monday 9/10/2018}
\newcommand{\pbStatement}[2]{
  \colorbox{yellow}{\parbox[t]{0.93\textwidth}
        {\bf (#1 \ifthenelse{#1 > 1}{points}{point})
 #2}}}
\newcommand{\fillIn}[1]{\fbox{\parbox[t]{0.93\textwidth}{\it #1}}}
\newcommand{\header}{
\noindent{\textit{CS 321 - Algorithms - Fall 2018\hfill \yourName}}
\begin{center}
  \textbf{\Large Assignment \assignNum}\\\colorbox{red}{\Large\bf
     \textcolor{white}{Due BEFORE 8:00AM on  \dueDate}}\\
  On time /  20\% off / no credit\\\textbf{Total points: 85}
  \bigskip

  \hrule\medskip This is an {\bf individual} assignment. You must work alone and
  submit your own work.\medskip \hrule \end{center}

\medskip
This assignment will test your knowledge from CS 212 and CS 271.  You
must write up your solutions to this assignment IN THIS FILE using
LaTeX by filling in all of the boxes below. If your submitted
\texttt{.tex} file does not compile, then you will receive 0 points.

All of the proofs that you write for this and subsequent assignments should be
{\bf as detailed as possible}.  You must write EACH step on its own line so that
it is very easy to see how each step follows directly from the previous step.
Furthermore, be sure to explain/justify each step in your proof. For example,
state that you added 5 to each side, that you simplified a fraction, etc. Do NOT
skip a bunch of algebraic steps in your proofs. For example, do not simply say
$\frac{k(k+1)}{2} + (k + 1) = \frac{(k+1)(k+2)}{2}$. It is a true statement, but
it is not trivial. It is not obvious that the left hand side equals the right
hand side. If you are unsure whether a step is needed or not, ask me first
before omitting it. Use the symbol $\square$ (i.e., \texttt{$\backslash$qed} in
your \texttt{.tex} file) to denote the end of each proof.

Here is an example of proper justification of algebraic steps and
formatting of a sequence
of equations:
\begin{align*}
  \frac{k(k+1)}{2} + (k + 1) & = \frac{k(k+1)}{2} + \frac{2(k + 1)}{2}
                                       & \mathrm{unified\ denominators}\\
                             & = \frac{k(k+1)+2(k+1)}{2}
                                       & \mathrm{added\ fractions}\\
                             & = \frac{(k+1)(k+2)}{2}
                             & \mathrm{factored\ out\ common\ factor}\\
\end{align*}\vspace*{-10mm}

Note how this list of equations is aligned vertically into three columns around
the equal signs and with the justifications on the right. This is the REQUIRED
format for your submissions whenever applicable.

You should NOT add any LaTeX packages to your \texttt{.tex} file. Note that I
included the \texttt{qtree} package and included its documentation in the
handout. You must use this package to produce your trees for problem~7.

\medskip
\textbf{Submission procedure:}
\begin{enumerate}[itemsep=-2mm]

\item Complete this file, called \texttt{a1.tex}, with your full name and
answers typed up below.
\item Compile this file to produce a file called \texttt{a1.pdf}. Make
  sure that this file compiles properly and that its contents and
  appearance meet the requirements described in this handout.
\item Create a directory called \texttt{a1} and copy exactly two files into this
directory, namely:\vspace*{-4mm}
  \begin{itemize}[itemsep=-1mm]
  \item \texttt{a1.tex} (this file with all of your answers added)
  \item \texttt{a1.pdf} (the compiled version of the file above)
  \end{itemize}
\item Zip up this directory to yield a file called \texttt{a1.zip}
\item Submit this zip file to the D2L dropbox for A1 before the deadline above.
\item Submit a single-sided, hard copy of your \texttt{a1.pdf} file BEFORE the
beginning of class on the due date above. \end{enumerate}

}

%************************************
% No need to modify anything above this line
% but DO fill this in!

\newcommand{\yourName}{\fbox{Your name goes here}}

%************************************g

\begin{document}
\header
\textbf{Problem statements}
\begin{enumerate}

  %%%%%%%%%%%%%%%%%%%%%%%%%%%%%%%%%%%%%%%%%%%%%%%%%%%%%%%%%%%%%%%%%%%%%%%%%%%
   \item                            % Problem 1
  %%%%%%%%%%%%%%%%%%%%%%%%%%%%%%%%%%%%%%%%%%%%%%%%%%%%%%%%%%%%%%%%%%%%%%%%%%%
  \pbStatement{10}{Write a direct proof of the following statement:\\
   \centerline{for all $n \in \mathbb{N}$, if $n$ is odd, then $n^3$ is odd}}

   {\bf Direct proof:}

  %%%%%%%%%%%%%%%%%%%%%%%%%%%%%%%%%%%%%%%%%%%%%%%%%%%%%%%%%%%%%%%%%%%%%%%%%%%
  \fillIn{Your proof goes here} % replace this line with your proof
  %%%%%%%%%%%%%%%%%%%%%%%%%%%%%%%%%%%%%%%%%%%%%%%%%%%%%%%%%%%%%%%%%%%%%%%%%%%

  %%%%%%%%%%%%%%%%%%%%%%%%%%%%%%%%%%%%%%%%%%%%%%%%%%%%%%%%%%%%%%%%%%%%%%%%%%%
  \item                            % Problem 2
  %%%%%%%%%%%%%%%%%%%%%%%%%%%%%%%%%%%%%%%%%%%%%%%%%%%%%%%%%%%%%%%%%%%%%%%%%%%
    \pbStatement{10}{State and disprove the contrapositive of the
      following statement:

  \centerline{for all $m, n \in \mathbb{N}$, if $mn < 361$, then $m < 19$ and
  $n < 19$}}

  {\bf Contrapositive:}

  %%%%%%%%%%%%%%%%%%%%%%%%%%%%%%%%%%%%%%%%%%%%%%%%%%%%%%%%%%%%%%%%%%%%%%%%%%%
  \fillIn{Write the contrapositive here} % replace this line with your statement

   %%%%%%%%%%%%%%%%%%%%%%%%%%%%%%%%%%%%%%%%%%%%%%%%%%%%%%%%%%%%%%%%%%%%%%%%%%%

  {\bf Proof:}

  %%%%%%%%%%%%%%%%%%%%%%%%%%%%%%%%%%%%%%%%%%%%%%%%%%%%%%%%%%%%%%%%%%%%%%%%%%%
  \fillIn{Your proof goes here}
  %%%%%%%%%%%%%%%%%%%%%%%%%%%%%%%%%%%%%%%%%%%%%%%%%%%%%%%%%%%%%%%%%%%%%%%%%%%

   %%%%%%%%%%%%%%%%%%%%%%%%%%%%%%%%%%%%%%%%%%%%%%%%%%%%%%%%%%%%%%%%%%%%%%%%%%%
   \item                            % Problem 3
   %%%%%%%%%%%%%%%%%%%%%%%%%%%%%%%%%%%%%%%%%%%%%%%%%%%%%%%%%%%%%%%%%%%%%%%%%%%

     \pbStatement{10}{Write a proof by contradiction of the following statement:

       \centerline{for all $m, n, p \in \mathbb{N}$, if $m^2 + n^2 = p^2$, then
       at least one of $m$ and $n$ is even}}

     {\bf Proof by contradiction:}

  %%%%%%%%%%%%%%%%%%%%%%%%%%%%%%%%%%%%%%%%%%%%%%%%%%%%%%%%%%%%%%%%%%%%%%%%%%%
  \fillIn{Your proof goes here}
  %%%%%%%%%%%%%%%%%%%%%%%%%%%%%%%%%%%%%%%%%%%%%%%%%%%%%%%%%%%%%%%%%%%%%%%%%%%

  %%%%%%%%%%%%%%%%%%%%%%%%%%%%%%%%%%%%%%%%%%%%%%%%%%%%%%%%%%%%%%%%%%%%%%%%%%%
  \item                            % Problem 4
  %%%%%%%%%%%%%%%%%%%%%%%%%%%%%%%%%%%%%%%%%%%%%%%%%%%%%%%%%%%%%%%%%%%%%%%%%%%

  \pbStatement{10}{Prove by induction that $P(n)$ holds for $n\geq 2$, where
  $P(n)$ is:

    \[
    \prod_{i=2}^{n} \left(1 - \frac{1}{i^2}\right) =  \frac{n+1}{2n}
    \]
  }

  {\bf Proof:}

  %%%%%%%%%%%%%%%%%%%%%%%%%%%%%%%%%%%%%%%%%%%%%%%%%%%%%%%%%%%%%%%%%%%%%%%%%%%
  \fbox{\textit{Your proof goes here}}
  %%%%%%%%%%%%%%%%%%%%%%%%%%%%%%%%%%%%%%%%%%%%%%%%%%%%%%%%%%%%%%%%%%%%%%%%%%%

  %%%%%%%%%%%%%%%%%%%%%%%%%%%%%%%%%%%%%%%%%%%%%%%%%%%%%%%%%%%%%%%%%%%%%%%%%%%
  \item                            % Problem 5
  %%%%%%%%%%%%%%%%%%%%%%%%%%%%%%%%%%%%%%%%%%%%%%%%%%%%%%%%%%%%%%%%%%%%%%%%%%%

  \pbStatement{10}{
  If $F_n$ is defined as follows:
  \[ F_n = \begin{cases}
    1 & n = 1 \\
    1 & n = 2 \\
    F_{n-1} + F_{n-2} & n\geq 3
   \end{cases}
  \]

  prove by induction that $P(n)$ holds for $n\geq 1$, where $P(n)$ is:

  \[
  \sum_{i=1}^{n} F^2_i =  F_n \cdot F_{n+1}
  \]
  }

  {\bf Proof:}

  %%%%%%%%%%%%%%%%%%%%%%%%%%%%%%%%%%%%%%%%%%%%%%%%%%%%%%%%%%%%%%%%%%%%%%%%%%%
  \fbox{\textit{Your proof goes here}}
  %%%%%%%%%%%%%%%%%%%%%%%%%%%%%%%%%%%%%%%%%%%%%%%%%%%%%%%%%%%%%%%%%%%%%%%%%%%

  %%%%%%%%%%%%%%%%%%%%%%%%%%%%%%%%%%%%%%%%%%%%%%%%%%%%%%%%%%%%%%%%%%%%%%%%%%%
   \item                            % Problem 6
  %%%%%%%%%%%%%%%%%%%%%%%%%%%%%%%%%%%%%%%%%%%%%%%%%%%%%%%%%%%%%%%%%%%%%%%%%%%

  \pbStatement{10}{ Let $S$ be the subset of
    $\mathbb{Z}\times\mathbb{Z}$ defined recursively/inductively by
    the following rules:
    \begin{description}
    \item [$R_0$:] $(0,0)$ is in $S$
    \item [$R_1$:] If $(a,b)$ is in $S$, then $(a,b+1)$ is also in $S$
    \item [$R_2$:] If $(a,b)$ is in $S$, then $(a+1,b+1)$ is also in $S$
    \item [$R_3$:] If $(a,b)$ is in $S$, then $(a+2,b+1)$ is also in $S$
    \item [$R_4$:] $S$ only contains elements that are generated from
      rules $R_0$ through $R_3$ above
    \end{description}

  Prove, using structural induction, that $\forall (a,b) \in S$,  $a\leq 2b$.
  }

  {\bf Proof:}

  %%%%%%%%%%%%%%%%%%%%%%%%%%%%%%%%%%%%%%%%%%%%%%%%%%%%%%%%%%%%%%%%%%%%%%%%%%%
  \fillIn{Your proof goes here}
  %%%%%%%%%%%%%%%%%%%%%%%%%%%%%%%%%%%%%%%%%%%%%%%%%%%%%%%%%%%%%%%%%%%%%%%%%%%

  %%%%%%%%%%%%%%%%%%%%%%%%%%%%%%%%%%%%%%%%%%%%%%%%%%%%%%%%%%%%%%%%%%%%%%%%%%%
   \item                            % Problem 7
  %%%%%%%%%%%%%%%%%%%%%%%%%%%%%%%%%%%%%%%%%%%%%%%%%%%%%%%%%%%%%%%%%%%%%%%%%%%

  \pbStatement{10}{Insert the following elements, in this order, into an AVL
  tree: 5, 20, 2, 7, 13, 4, 22 to be able to answer the following questions.}

    \pbStatement{7}{[Part a] Show the tree after each insertion. For each step,
    if a rotation is needed, you must show the tree both before and after the
    rotation. In all other steps, only one tree must be included.}


 \begin{itemize}
   \item Insert 5:

  %%%%%%%%%%%%%%%%%%%%%%%%%%%%%%%%%%%%%%%%%%%%%%%%%%%%%%%%%%%%%%%%%%%%%%%%%%%
  \fillIn{Your tree (or trees) goes here}
  %%%%%%%%%%%%%%%%%%%%%%%%%%%%%%%%%%%%%%%%%%%%%%%%%%%%%%%%%%%%%%%%%%%%%%%%%%%

   \item Insert 20:

  %%%%%%%%%%%%%%%%%%%%%%%%%%%%%%%%%%%%%%%%%%%%%%%%%%%%%%%%%%%%%%%%%%%%%%%%%%%
  \fillIn{Your tree (or trees) goes here}
  %%%%%%%%%%%%%%%%%%%%%%%%%%%%%%%%%%%%%%%%%%%%%%%%%%%%%%%%%%%%%%%%%%%%%%%%%%%

   \item Insert 2:

  %%%%%%%%%%%%%%%%%%%%%%%%%%%%%%%%%%%%%%%%%%%%%%%%%%%%%%%%%%%%%%%%%%%%%%%%%%%
  \fillIn{Your tree (or trees) goes here}
  %%%%%%%%%%%%%%%%%%%%%%%%%%%%%%%%%%%%%%%%%%%%%%%%%%%%%%%%%%%%%%%%%%%%%%%%%%%

   \item Insert 7:

  %%%%%%%%%%%%%%%%%%%%%%%%%%%%%%%%%%%%%%%%%%%%%%%%%%%%%%%%%%%%%%%%%%%%%%%%%%%
  \fillIn{Your tree (or trees) goes here}
  %%%%%%%%%%%%%%%%%%%%%%%%%%%%%%%%%%%%%%%%%%%%%%%%%%%%%%%%%%%%%%%%%%%%%%%%%%%

   \item Insert 13:

  %%%%%%%%%%%%%%%%%%%%%%%%%%%%%%%%%%%%%%%%%%%%%%%%%%%%%%%%%%%%%%%%%%%%%%%%%%%
  \fillIn{Your tree (or trees) goes here}
  %%%%%%%%%%%%%%%%%%%%%%%%%%%%%%%%%%%%%%%%%%%%%%%%%%%%%%%%%%%%%%%%%%%%%%%%%%%

   \item Insert 4:

  %%%%%%%%%%%%%%%%%%%%%%%%%%%%%%%%%%%%%%%%%%%%%%%%%%%%%%%%%%%%%%%%%%%%%%%%%%%
  \fillIn{Your tree (or trees) goes here}
  %%%%%%%%%%%%%%%%%%%%%%%%%%%%%%%%%%%%%%%%%%%%%%%%%%%%%%%%%%%%%%%%%%%%%%%%%%%

   \item Insert 22:

  %%%%%%%%%%%%%%%%%%%%%%%%%%%%%%%%%%%%%%%%%%%%%%%%%%%%%%%%%%%%%%%%%%%%%%%%%%%
  \fillIn{Your tree (or trees) goes here}
  %%%%%%%%%%%%%%%%%%%%%%%%%%%%%%%%%%%%%%%%%%%%%%%%%%%%%%%%%%%%%%%%%%%%%%%%%%%

\end{itemize}

\pbStatement{1}{[Part b] What is the depth of node 4 in the final tree?}

The depth of node 4 is :

\fillIn{your answer goes here}

\pbStatement{1}{[Part c] Write down, on one line, all of the node
     values in the preorder traversal of this tree.}

Preorder traversal:

\fillIn{your answer goes here}

\pbStatement{1}{[Part d] Write down, on one line, all of the node
  values in the postorder traversal of this tree.}

Postorder traversal:

\fillIn{your answer goes here}

  %%%%%%%%%%%%%%%%%%%%%%%%%%%%%%%%%%%%%%%%%%%%%%%%%%%%%%%%%%%%%%%%%%%%%%%%%%%
   \item                            % Problem 8
  %%%%%%%%%%%%%%%%%%%%%%%%%%%%%%%%%%%%%%%%%%%%%%%%%%%%%%%%%%%%%%%%%%%%%%%%%%%

\pbStatement{5}{Insert the following elements, in this order, into a
  max heap:

  5, 20, 2, 7, 13, 4, 22

  Show all of the intermediate steps,
  i.e., show the heap after each element is inserted, by filling in the
  following table, in which each row represents the 1-indexed heap array
  after an additional insertion.}

\fillIn{Write down your complete answer in the following table making sure that
  empty array locations remain empty cells in the table.}

\begin{tabular}{l|c|c|c|c|c|c|c|}
          & 1  & 2  & 3  & 4  & 5  & 6  & 7  \\ \hline
insert 5  &    &    &    &    &    &    &    \\ \hline
insert 20 &    &    &    &    &    &    &    \\ \hline
insert 2  &    &    &    &    &    &    &    \\ \hline
insert 7  &    &    &    &    &    &    &    \\ \hline
insert 13 &    &    &    &    &    &    &    \\ \hline
insert 4  &    &    &    &    &    &    &    \\ \hline
insert 22 &    &    &    &    &    &    &    \\ \hline
\end{tabular}

  %%%%%%%%%%%%%%%%%%%%%%%%%%%%%%%%%%%%%%%%%%%%%%%%%%%%%%%%%%%%%%%%%%%%%%%%%%%
  \item                            % Problem 9
  %%%%%%%%%%%%%%%%%%%%%%%%%%%%%%%%%%%%%%%%%%%%%%%%%%%%%%%%%%%%%%%%%%%%%%%%%%%

\pbStatement{5}{Let $G = (V,E,w)$ be an undirected graph, with vertex set $V =
\{1, 2, 3, 4, 5, 6,$ $7 \}$, edge set $E = \{(1, 2), (1, 4), (1, 6), (2, 3), (2,
4), (3,4), (3, 5), (4,5), (4, 6), (4, 7), (5, 7), (6, 7)\}$ and weight function
$w : E \rightarrow \mathbb{N}$, such that $w(1, 4) = 6$, $w(1, 6) = 8$, $w(2, 1)
= 1$, $w(2, 3) = 4$, $w(2, 4) = 5$, $w(3, 4) = 5$, $w(3, 5) = 2$, $w(4, 5) = 9$,
$w(4, 6) = 6$, $w(4, 7) = 6$, $w(5, 7) = 3$, and $w(6, 7) = 7$. What is the
total edge weight of a minimal spanning tree of $G$? }

\fillIn{Your answer goes here}

  %%%%%%%%%%%%%%%%%%%%%%%%%%%%%%%%%%%%%%%%%%%%%%%%%%%%%%%%%%%%%%%%%%%%%%%%%%%
  \item                            % Problem 10
  %%%%%%%%%%%%%%%%%%%%%%%%%%%%%%%%%%%%%%%e%%%%%%%%%%%%%%%%%%%%%%%%%%%%%%%%%%%

\pbStatement{5}{Let $G = (V,E)$ be a DIRECTED graph, where $V$ and
  $E$ are from the previous problem and the direction of the edges is
  given by the order of the vertices in the weight function in the
  previous problem (e.g., $w(2,1)$ means that the edge between
  vertices 1 and 2 is directed from 2 to 1).  Does $G$ have a
  topological ordering? If so, provide such an ordering.  If not,
  explain why not.  }

\fillIn{Your answer goes here}

\end{enumerate}
\end{document}
